\chapter [Introduction to Analog Communication]{Introduction to Analog Communication}


\cite{ACmanual}Communication is the transfer of information from one place to another. Radio communication uses electrical energy to transmit information. 

The transmitted information is the \textbf {intelligence signal} or \textbf{ message signal}.
 Message signals are in the \textbf {Audio Frequency (AF)} range of low frequencies from about 20 Hz to 20 kHz.
 
 
The \textbf{Radio Frequency (RF)}  is the carrier signal. Carrier signals have high frequencies that range from 10 kHz up to about 1000 GHz.
A radio transmitter sends the low frequency message signal at the higher carrier signal frequency by combining the message signal with the carrier signal.

\textbf{Modulation} is the process of changing a characteristic of the carrier signal with the message
signal. In the transmitter, the message signal modulates the carrier signal.
The modulated carrier signal is sent to the receiver where \textbf{demodulation} of the carrier occurs to
recover the message signal.\\[10pt]
\textsc{\textbf {IMPORTANT TERMS}}
\begin{itemize}

\item \textbf{Electromagnetic waves} - the radiant energy produced by oscillation of an electric
charge.

\item \textbf{Message signal} - any signal that contains information; it is also called
the intelligence signal. 
\item \textbf{Audio Frequency (AF)} - frequencies that a person can hear.
AF signals range from about 20 Hz to 20 kHz.
\item \textbf{Radio Frequency (RF)} - the transmission frequency of electromagnetic (radio) signals.
RF frequencies are from about 300 kHz to the 1,000,000 kHz range.
\item \textbf{Carrier signal} - a single, high-frequency signal that can be modulated by a message
signal and transmitted.
\item \textbf{Modulation} - the process of combining the message signal with the carrier signal that
causes the message signal to vary a characteristic of the carrier signal.
\item \textbf{Demodulation} - the process of recovering or detecting the message signal from the
modulated carrier frequency.
\item \textbf{Amplitude Modulation (AM)} - the process of combining the message signal with
the carrier signal and the two sidebands: the lower sideband and the upper
sideband.
\item \textbf{Frequency Modulation (FM)} - the process of combining the message signal with
the carrier signal that causes the message signal to vary the frequency of the
carrier signal.
\item \textbf{Phase Modulation (PM)} - the process of combining the message signal with the
carrier signal that causes the message signal to vary the phase of the carrier signal.
\item \textbf{Angle modulation} - the process of combining the message signal with the carrier signal that causes the message signal to vary the frequency and/or phase of the
carrier signal.
\item \textbf{Balanced modulator} - an amplitude modulator that can be adjusted to control the
amount of modulation.
\item \textbf{Double-Sideband (DSB)} - an amplitude modulated signal in which the carrier is
suppressed, leaving only the two sidebands: the lower sideband and the upper
sideband.
\item \textbf{Mixer}- an electronic circuit that combines two frequencies.
\item \textbf{Phase detector} - an electronic circuit whose output varies with the phase differential
of the two input signals.
\item \textbf{Envelopes}- the waveform of the amplitude variations of an amplitude modulated
signal. 
\item \textbf{Sidebands} - the frequency bands on each side of the carrier frequency that
are formed during modulation; the sideband frequencies contain the intelligence of
the message signal.
\item \textbf{AM} - an amplitude modulated signal that contains the carrier signal and the two
sidebands: the lower sideband and the upper sideband.
\item \textbf{Bandwidth} - the frequency range, in hertz (Hz), between the upper and lower
frequency limits. 
\item \textbf{Harmonics} - signals with frequencies that are an integral multiple of
the fundamental frequency. 

\end {itemize}
